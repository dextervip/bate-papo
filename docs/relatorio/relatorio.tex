\documentclass{abnt}
%\usepackage[a4paper, inner=1.5cm, outer=2cm, top=3cm, bottom=2cm, bindingoffset=1cm]{geometry}
\usepackage[utf8]{inputenc}
\usepackage[english,brazilian]{babel}
\usepackage{rotating}
\usepackage{hyperref}
\usepackage{url}
\usepackage{indentfirst}
\hypersetup{%
    pdfborder = {0 0 0}
}
\usepackage{graphics}
\graphicspath{{./figuras/}}
\usepackage{placeins}
%\usepackage{lscape}
\usepackage{pdflscape}

\begin{document}

\autor{Rafael Tavares Amorim \par Rafhael Rodrigues Cunha\par Marcelo Maia Lopes }

\titulo{Redes e Sistemas Distribuídos}

%\orientador{Prof.}

\instituicao{Universidade Federal do Pampa \par Engenharia da Software}

\local{Alegrete - RS, Brasil}

\data{05 de Maio de 2012}

\capa

\folhaderosto

\tableofcontents


\chapter{INTRODUÇÃO}
Esse Documento tem por objetivo auxiliar o usuário no manuseio do chat, oferecendo ao mesmo um manual do usuário contendo manual de Instalação, configuração e utilização do software.

O documento também contem um manual de desenvolver, para fins de correção do trabalho e curiosidades, possuindo assim diagrama de classes, explicações sucintas sobre a linguagem que foi utilizada, e por fim uma introdução sobre o protocolo usado e o motivo pela escolha do mesmo.

\clearpage
\chapter{Manual do Usuário}

	\section{Instação}	
	Como apoio a instalação será enviado juntamente com o relatorio os arquivos fontes do projeto e os arquivos já compilados (jar); Para instalar basta apenas copiar os arquivos jars (bate-papo-cliente.jar, bate-papo-servidor.jar) para a pasta desejada, abrir o terminal do seu sistema operacional, como exemplo utilizarei o windows, e fazer os seguintes procedimentos:
	\begin{enumerate}
	
	\item  Navegar ate a pasta aonde você copiou os jars, tanto o bate-papo-servidor quanto o bate-papo-cliente.
	Lembrando que para navegação utilizamos: 
	
	\begin{itemize}
	\item comando 'cd' - para abrir uma pasta;
	\item comando 'ls' - para visualizar o que esta dentro de uma pasta;
	\end{itemize}	
	
	\item Quando você chegar até a pasta aonde se encontra esses arquivos, digite java -jar nomedoarquivo.jar
	no nosso exemplo então ficaria da seguinte maneira:
	\begin{itemize}
	\item java - jar bate-papo-servidor.jar
	\end{itemize}
	Logo após darmos esse comando será inicializado o nosso servidor no qual receberá as mensagens do chat.
	
	\item Após, seguimos no terminal e dessa vez digitamos o seguinte comando: java -jar bate-papo-cliente.jar
	desse modo iniciaremos a interface gráfica na qual vai ser possível começar a utilização do chat.
	\end{enumerate}
	
	\section{Configuração}		
	\section{Utilização}		
		


\clearpage
\chapter{Manual do Desenvolvedor}
\section{Diagrama de Classes}	

\begin{figure}[htp]
\begin{center}
\includegraphics[width=300px]{jbpm_overview}
\caption{Diagrama de Classes}
\label{fig:jbpmOverview}
\end{center}
\end{figure}
\FloatBarrier

\section{Linguagem Utilizada}	
A linguagem utilizada de apoio para desenvolvimento da solução do trabalho proposto foi o java, porque alem de ele ser multiplataforma,ou seja, rodar em diversos sistemas operacionais ele tambem possui muitos tratamentos na parte de sockets que é uma classe utilizada na implementação da solução, na parte de comunicação entre um cliente e outro.

\section{Protocolo}		





\clearpage
%Referências Bibliograficas
\nocite{*}
\bibliographystyle{plain}		
\bibliography{bibliografia}		
\end{document}